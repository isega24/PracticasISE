%%%%%%%%%%%%%%%%%%%%%%%%%%%%%%%%%%%%%%%%%
% Short Sectioned Assignment LaTeX Template Version 1.0 (5/5/12)
% This template has been downloaded from: http://www.LaTeXTemplates.com
% Original author:  Frits Wenneker (http://www.howtotex.com)
% License: CC BY-NC-SA 3.0 (http://creativecommons.org/licenses/by-nc-sa/3.0/)
%%%%%%%%%%%%%%%%%%%%%%%%%%%%%%%%%%%%%%%%%

%----------------------------------------------------------------------------------------
%	PACKAGES AND OTHER DOCUMENT CONFIGURATIONS
%----------------------------------------------------------------------------------------

\documentclass[paper=a4, fontsize=11pt]{scrartcl} % A4 paper and 11pt font size

% ---- Entrada y salida de texto -----

\usepackage[T1]{fontenc} % Use 8-bit encoding that has 256 glyphs
\usepackage[utf8]{inputenc}
%\usepackage{fourier} % Use the Adobe Utopia font for the document - comment this line to return to the LaTeX default

% ---- Idioma --------

\usepackage[spanish, es-tabla]{babel} % Selecciona el español para palabras introducidas automáticamente, p.ej. "septiembre" en la fecha y especifica que se use la palabra Tabla en vez de Cuadro

% ---- Otros paquetes ----

\usepackage{amsmath,amsfonts,amsthm} % Math packages
%\usepackage{graphics,graphicx, floatrow} %para incluir imágenes y notas en las imágenes
\usepackage{graphics,graphicx, float, url} %para incluir imágenes y colocarlas

% Para hacer tablas comlejas
%\usepackage{multirow}
%\usepackage{threeparttable}
\usepackage{float}


%\usepackage{sectsty} % Allows customizing section commands
%\allsectionsfont{\centering \normalfont\scshape} % Make all sections centered, the default font and small caps

\usepackage{fancyhdr} % Custom headers and footers
\pagestyle{fancyplain} % Makes all pages in the document conform to the custom headers and footers
\fancyhead{} % No page header - if you want one, create it in the same way as the footers below
\fancyfoot[L]{} % Empty left footer
\fancyfoot[C]{} % Empty center footer
\fancyfoot[R]{\thepage} % Page numbering for right footer
\renewcommand{\headrulewidth}{0pt} % Remove header underlines
\renewcommand{\footrulewidth}{0pt} % Remove footer underlines
\setlength{\headheight}{13.6pt} % Customize the height of the header

\numberwithin{equation}{section} % Number equations within sections (i.e. 1.1, 1.2, 2.1, 2.2 instead of 1, 2, 3, 4)
\numberwithin{figure}{section} % Number figures within sections (i.e. 1.1, 1.2, 2.1, 2.2 instead of 1, 2, 3, 4)
\numberwithin{table}{section} % Number tables within sections (i.e. 1.1, 1.2, 2.1, 2.2 instead of 1, 2, 3, 4)

\setlength\parindent{0pt} % Removes all indentation from paragraphs - comment this line for an assignment with lots of text

\newcommand{\horrule}[1]{\rule{\linewidth}{#1}} % Create horizontal rule command with 1 argument of height


\title{	
	\normalfont \normalsize 
	\textsc{{\bf Ingeniería de Servidores (2014-2015)} \\ Grado en Ingeniería Informática \\ Universidad de Granada} \\ [25pt] % Your university, school and/or department name(s)
	\horrule{0.5pt} \\[0.4cm] % Thin top horizontal rule
	\huge Memoria Práctica 2 \\ % The assignment title
	\horrule{2pt} \\[0.5cm] % Thick bottom horizontal rule
}

\author{Iván Sevillano García} % Nombre y apellidos

\date{\normalsize\today} % Incluye la fecha actual

\begin{document}

\maketitle % Muestra el Título

\newpage %inserta un salto de página

\tableofcontents % para generar el índice de contenidos

\newpage

\section{Instalación de servicios y configuraciones.}
En esta sección se responden a las preguntas referentes a la administración y configuración de software en los distintos sistemas operativos.
\subsection{Yum. Gestor de paquetes de CentOS.}

Esta aplicación es el gestor predeterminado de CentOS, Red Hat, Fedora y derivados.

\begin{itemize}
	\item \textbf{Liste los argumentos de yum necesarios para instalar, buscar y eliminar paquetes.}\\
	Para instalar un paquete del que ya sabemos el nombre podremos utilizar yum seguido del comando \textbf{install} tras el que podremos el paquete a instalar. Para buscar, el comando \textbf{search} seguido de alguna referencia indirecta del paquete(parte del nombre en general) nos dará como resultado el paquete que estemos buscando. Para eliminar paquetes, yum tiene comandos \textbf{remove} y \textbf{erase} seguidos del paquete a eliminar, lo que produce redundancia de funciones, ya que hacen exactamente el mismo trabajo. Esto es un fallo de diseñoy se deberá eliminar un o de estos términos en un futuro.\\
	
	\item \textbf{¿Que hay que hacer para que yum tenga acceso a internet en el aula de prácticas(puesto que esta utiliza un proxy por detrás)?}\\
	Para contestar a esta pregunta se nos ofrece dos pistas: acceder al archivo de configuración de yum en /etc/yum/ y que el proxy a utilizar es \textbf{stargate.ugr.es:3128}. Buscamos en el manual de configuración de yum y nos dice que necesitamos incluir en el archivo \textbf{yum.conf} la URL del proxy. También hay opciones para acceder al proxy con un usuario y una contraseña(proxy\_username y proxy\_password).
	
\end{itemize}
Para responder a estas preguntas, hemos consultado el manual que yum nos ofrece\cite{a1} y la página referente a el archivo de configuración\cite{a2}.

\subsection{Apt. Gestor de paquetes de Debian.}
Esta aplicación es el gestor predeterminado de Debian y sus derivados, como por ejemplo ubuntu.
\begin{itemize}
	\item \textbf{Indique el comando para buscar un paquete en un repositorio y el
		correspondiente para instalarlo.}\\
	Para buscar un paquete del que tenemos parte del nombre o similar, debemos utilizar el comando \textbf{apt search}, y para instalarle, el comando \textbf{apt install}.\cite{b1}
	
	
	 \item \textbf{Indique qué ha modificado para que apt pueda acceder a los servidores de paquetes a través del proxy. ¿Cómo añadimos un nuevo repositorio?}\\
	 Según el manual proporcionado por apt.conf:\\
	 
	 "\textit{http::Proxy define el proxy predeterminado que utilizar para direcciones HTTP URI. Utiliza el formato estándar
	 	http://[[usuario][:contraseña]@]máquina[:puerto]/. También se puede especificar un proxy por cada máquina usando la forma
	 	http::Proxy::<máquina> con la palabra especial DIRECT que significa que no se use ningún proxy. La variable de entorno http\_proxy
	 	se usará en caso de no definir ninguna de las opciones anteriores.
	 }"\cite{b2}\\
	 
	 También según el mismo manual, la forma de añadir repositorios es añadirlos al archivo /etc/apt/sources.list con permisos de root. Se puede utilizar el comando edit-sources, con el mismo resultado que editarlos con un editor de texto directamente, ya que este comando solo llama a un editor a escoger\cite{b1}.
	 
	 
	 
\end{itemize}

\subsection{Cuestión opcional: Gestor de paquetes de OpenSUSE.}
Por cuestión de tiempo, en esta asignatura no utilizaremos OpenSUSE, pero si vamos a responder a la pregunta de cuál es el gestor de paquetes de este Sistema Operativo.\\

En la página oficial de OpenSUSE, en la parte que explica el funcionamiento de su gestión de paquetes\cite{c1} se dice que éste tiene dos gestores, uno con un entorno gráfico y otro para consola. El primero es YaST y el segundo, Zypper.

\newpage
\section{Instalación del servicio de acceso seguro(SSH)}
 
 Para el acceso remoto desde una máquina a otra hay distintos protocolos. Entre ellos, destacan telnet y ssh. Sin embargo, para las conexiones que realizaremos, utilizaremos ssh.
 
 \begin{itemize}
 	\item ¿Por qué?\\
 	Telnet es un servicio básico de conexión entre máquinas, en el cuál sólo hay envío y recepción de mensajes. Ssh (Secure shell), además de esto, los mensajes se envían de forma segura, es decir, encriptando mensajes, usando huellas digitales para comprobar de donde vienen los mensajes y demás algoritmos de seguridad. En definitiva, ssh no tiene nada que ver con telnet por su carácter inseguro.
 \end{itemize}
 
 Tras instalar el servicio openssh-server en la máquina, probaremos ahora a conectarnos remotamente. Para ello, tenemos que saber


 


\newpage
\begin{thebibliography}{xx}
	\bibitem{a1} http://linux.die.net/man/8/yum
	\bibitem{a2} http://linux.die.net/man/5/yum.conf
	\bibitem{b1} http://linux.die.net/man/8/apt
	\bibitem{b2} http://linux.die.net/man/5/apt.conf
	
\end{thebibliography}
\end{document}